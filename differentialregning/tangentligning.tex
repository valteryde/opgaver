\documentclass[]{article}
\usepackage{graphicx}
\usepackage{xcolor}
\newcounter{spgcounter}
\newenvironment{question}[2]{\addtocounter{spgcounter}{1} SPØRGSMÅL \thespgcounter\\}{\hspace{50px}}
\newcommand{\name}[1]{{\huge #1}\\}
\newcommand{\tag}[1]{#1}
\newcommand{\cover}[1]{\includegraphics[width=.7\linewidth]{#1}\\}
\newcommand{\image}[1]{\includegraphics[width=.5\linewidth]{#1}\\}
\newcommand{\answer}[1]{{\color{green} SVAR: #1}\\}
\newcommand{\alt}[1]{{\color{red} ALT: #1}\\}

\begin{document}

\name{Tangentligning}
\cover{tangentligning.png}
\tag{Tangenter}
\tag{f'(x)}
\begin{question}{multi}

    Bestem tangentligningen til $f(x) = x^2$ i punktet $x = 1$.
    \answer{$y = 2x - 1$}
    \alt{$y = x^2$}
    \alt{$y = 2x + 1$}
    \alt{$y = x - 1$}
    \end{question}
    
    \begin{question}{multi}
    Bestem tangentligningen til $f(x) = e^x$ i punktet $x = 0$.
    \answer{$y = x + 1$}
    \alt{$y = e^x$}
    \alt{$y = x$}
    \alt{$y = x + e$}
    \end{question}
    
    \begin{question}{multi}
    Bestem tangentligningen til $f(x) = \ln(x)$ i punktet $x = 1$.
    \answer{$y = x - 1$}
    \alt{$y = x + 1$}
    \alt{$y = x \ln(1)$}
    \alt{$y = 2x - 1$}
    \end{question}
    
    \begin{question}{multi}
    Bestem tangentligningen til $f(x) = \cos(x)$ i punktet $x = \frac{\pi}{2}$.
    \answer{$y = -x + \frac{\pi}{2}$}
    \alt{$y = x$}
    \alt{$y = \sin(x)$}
    \alt{$y = -x + \pi$}
    \end{question}
    
    \begin{question}{multi}
    Bestem tangentligningen til $f(x) = -x^3$ i punktet $x = -1$.
    \answer{$y = -3x - 2$}
    \alt{$y = 3x + 1$}
    \alt{$y = -3x + 1$}
    \alt{$y = 3x - 2$}
    \end{question}
    
    \begin{question}{multi}
    Bestem tangentligningen til $f(x) = \sqrt{x}$ i punktet $x = 4$.
    \answer{$y = \frac{1}{4}x + 1$}
    \alt{$y = \frac{1}{2}x + 1$}
    \alt{$y = \frac{1}{4}x - 1$}
    \alt{$y = \frac{1}{2}x + 2$}
    \end{question}
        
    \begin{question}{multi}
    Bestem tangentligningen til $f(x) = \frac{1}{x}$ i punktet $x = 1$.
    \answer{$y = -x + 2$}
    \alt{$y = -x$}
    \alt{$y = \frac{1}{x} - 1$}
    \alt{$y = -x + 1$}
    \end{question}
    
\end{document}
