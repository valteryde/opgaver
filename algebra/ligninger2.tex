\documentclass[]{article}
\usepackage{graphicx}
\usepackage{xcolor}
\newcounter{spgcounter}
\newenvironment{question}[2]{\addtocounter{spgcounter}{1} SPØRGSMÅL \thespgcounter\\ }{\hspace{50px}}
\newcommand{\name}[1]{{\huge #1}\\}
\newcommand{\tag}[1]{#1}
\newcommand{\cover}[1]{\includegraphics[width=.7\linewidth]{#1}\\}
\newcommand{\image}[1]{\includegraphics[width=.5\linewidth]{#1}\\}
\newcommand{\answer}[1]{{\color{green} SVAR: #1}\\}
\newcommand{\alt}[1]{{\color{red} ALT: #1}\\}

\begin{document}

\name{Ligninger ii}
\cover{brøkkerligninger.png}
\tag{Uden hjælpemidler}
\tag{Brøkker}
\tag{1/x}

\begin{question}{multi}
Vis at løsningen til $\frac{1}{x} + 2 = 10$ er $x=\frac{1}{8}$
\answer{Aflever}
\end{question}

\begin{question}{multi}
Løs $\frac{1}{x} - 4 = 3$

\answer{$\frac{1}{7}$}
\alt{$\frac{2}{5}$}
\alt{$\frac{1}{7}$}
\alt{$\frac{-2}{7}$}
\end{question}

\begin{question}{multi}
Løs $\frac{1}{x+1} + 7 = 1$

\answer{$-\frac{7}{6}$}
\alt{$\frac{7}{6}$}
\alt{$\frac{2}{3}$}
\alt{$-\frac{2}{3}$}
\end{question}

\begin{question}{multi}
Løs $\frac{1}{x+2} \cdot 5 = 2$
\answer{$\frac{1}{2}$}
\alt{$-\frac{1}{2}$}
\alt{$-\frac{3}{1}$}
\alt{$-\frac{6}{4}$}
\end{question}

\begin{question}{multi}
Løs $\frac{2}{x+1} = x$
\answer{$-2$ og $1$}
\alt{$2$ og $-1$}
\alt{$3$ og $4$}
\alt{$-3$ og $-4$}

\end{question}

\end{document}
