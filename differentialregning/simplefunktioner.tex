\documentclass[]{article}
\usepackage{graphicx}
\usepackage{xcolor}
\newenvironment{question}[2]{SPØRGSMÅL\\}{\hspace{50px}}
\newcommand{\name}[1]{{\huge #1}\\}
\newcommand{\tag}[1]{#1}
\newcommand{\cover}[1]{\includegraphics[width=.7\linewidth]{#1}\\}
\newcommand{\image}[1]{\includegraphics[width=.5\linewidth]{#1}\\}
\newcommand{\answer}[1]{{\color{green} SVAR: #1}\\}
\newcommand{\alt}[1]{{\color{red} ALT: #1}\\}

\begin{document}

\name{Simple funktioner}
\cover{tangent.png}
\tag{Funktioner}
\tag{f'(x)}

\begin{question}{multi}\id{d20cf3ed30d54facaaf94a1f17bc06cb}
    
Hvad er $f'(x)$ når $f(x)=2\cdot x$

% \image{test.jpg}

\answer{2}
\alt{$2x$}
\alt{$x^2$}

\end{question}

\end{document}

% HUSK AT LAVE TIL EN ZIP FIL\documentclass[]{article}
\usepackage{graphicx}
\usepackage{xcolor}
\newenvironment{question}[2]{SPØRGSMÅL\\}{\hspace{50px}}
\newcommand{\image}[1]{\includegraphics[width=.5\linewidth]{#1}}
\newcommand{\answer}[1]{{\color{green} SVAR: #1}\\}
\newcommand{\alt}[1]{{\color{red} ALT: #1}\\}

\begin{document}

\begin{question}{multi}
	
	Hvad er $f'(x)$ når $f(x)=2\cdot x$
	
	\answer{$2$}
	\alt{$2x$}
	\alt{$x^2$}
	
	\end{question}
	
	\begin{question}{multi}
		
		Hvad er $f'(x)$ når $f(x)=10^5$
		
		\answer{$0$}
		\alt{1}
		\alt{2}
		\alt{$50^4$}
		
	\end{question}
	
	\begin{question}{multi}
		
		Hvad er $f'(x)$ når $f(x)=e^x$
		
		\answer{$e^x$}
		\alt{$2e^x$}
		\alt{$xe^x$}
		\alt{$0$}
		
	\end{question}
	
	\begin{question}{multi}
		
		Hvad er $f'(x)$ når $f(x)=3\cdot x^2$
		
		\answer{$6 x$}
		\alt{$3x$}
		\alt{$6x^2$}
		\alt{$x$}
		
	\end{question}
	
	\begin{question}{multi}
		
		Hvad er $f'(x)$ når $f(x)=\ln(x)$
		
		\answer{$\frac{1}{x}$}
		\alt{$x$}
		\alt{$0$}
		\alt{$1$}
		
	\end{question}
	
	\begin{question}{multi}
		
		Hvad er $f'(2)$ når $f(x)=\ln(x)$
		
		\answer{$\frac{1}{2}$}
		\alt{$\frac{1}{3}$}
		\alt{$1$}
		\alt{$\frac{1}{4}$}
		
	\end{question}
	
	\begin{question}{multi}
		
		Hvad betyder det for grafen at $f'(2)=5$
		
		\answer{Tangenthældningen i $x=2$ er 5}
		\alt{Grafen går igennem punktet (2, 5)}
		\alt{Grafen tangere i punktet (2, 5)}
		\alt{Grafen er konstant i $x=2$}
		
	\end{question}

\end{document}
