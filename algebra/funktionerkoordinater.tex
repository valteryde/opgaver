\documentclass[]{article}
\usepackage{graphicx}
\usepackage{xcolor}
\newcounter{spgcounter}
\newenvironment{question}[2]{\addtocounter{spgcounter}{1} SPØRGSMÅL \thespgcounter\\}{\hspace{50px}}
\newcommand{\name}[1]{{\huge #1}\\}
\newcommand{\tag}[1]{#1}
\newcommand{\cover}[1]{\includegraphics[width=.7\linewidth]{#1}\\}
\newcommand{\image}[1]{\includegraphics[width=.5\linewidth]{#1}\\}
\newcommand{\answer}[1]{{\color{green} SVAR: #1}\\}
\newcommand{\alt}[1]{{\color{red} ALT: #1}\\}


\begin{document}

\name{Funktioner i et koordinatsystem}
\cover{funktionergrafkoordinat.png}
\tag{Funktioner}
\tag{Koordinatsystem}

\begin{question}{multi}
    
Givet en funktion $f(x)=2 \cdot x$ og dens tilhørende graf

\image{linearfunktion.png}

Aflæs på grafen hvad $f(8)$ er.

\answer{16}
\alt{20}
\alt{12}
\alt{$-10$}

\end{question}

\begin{question}{multi}
    
    Aflæs på grafen hvad $f(x)=14$ er. Altså, hvad skal $x$ være når $f(x)$ skal være 14.
    
    \answer{7}
    \alt{5}
    \alt{-2}
    \alt{10}
    
\end{question}

\begin{question}{multi}
    
    Aflæs på grafen hvad $f(x)=4$ er.
    
    \answer{2}
    \alt{1}
    \alt{3}
    \alt{4}

\end{question}


\begin{question}{multi}

Hvad er $f(3)$

\answer{6}
\alt{3}
\alt{2}
\alt{$-6$}

\end{question}

\begin{question}{multi}
    
Hvad er $f(-1)$

\answer{$-2$}
\alt{$-3$}
\alt{2}
\alt{$1$}
\alt{$-1$}

\end{question}

\begin{question}{multi}
    
Hvilke af følgende ligninger er det samme som $f(x)=1$
    
\answer{$2x=1$}
\alt{$2\cdot 1 = 1$}
\alt{$2=4$}
\alt{$1 \cdot x=2$}
    
\end{question}

\begin{question}{multi}
    
Løs $f(x)=1$
        
\answer{$\frac{1}{2}$}
\alt{$1$}
\alt{$2$}
\alt{$\frac{2}{1}$}

\end{question}


\begin{question}{multi}
    
Givet nedenstående graf

\image{andengradsfunktionen1.png}

Aflæs hvad $f(3)$

\answer{$14$}
\alt{$20$}
\alt{$-12$}
\alt{$5$}

\end{question}

\begin{question}{multi}
    
Aflæs de to $x$-kordinater hvor $f(x)=10$
    
\answer{Omkring $-4,5$ og $2,5$}
\alt{Omkring $1$ og $10$}
\alt{Omkring $-5$ og $5$}

\end{question}

\end{document}
